%% ------------------------------------------------------------------------- %%
\chapter{Conclus�es}
\label{cap:conclusoes}

Texto texto texto texto texto texto texto texto texto texto texto texto texto
texto texto texto texto texto texto texto texto texto texto texto texto texto
texto texto texto texto texto texto\footnote{Exemplo de refer�ncia para p�gina
Web: \url{www.vision.ime.usp.br/~jmena/stuff/tese-exemplo}}.

%------------------------------------------------------
\section{Considera��es Finais} 

Texto texto texto texto texto texto texto texto texto texto texto texto texto
texto texto texto texto texto texto texto texto texto texto texto texto texto
texto texto texto texto texto texto. 

%------------------------------------------------------
\section{Sugest�es para Pesquisas Futuras} 

Texto texto texto texto texto texto texto texto texto texto texto texto texto
texto texto texto texto texto texto texto texto texto texto texto texto texto
texto texto texto texto texto texto.

Finalmente, leia o trabalho de Uri Alon \cite{alon09:how} no qual apresenta-se
uma reflex�o sobre a utiliza��o da Lei de Pareto para tentar definir/escolher
problemas para as diferentes fases da vida acad�mica.  A dire��o dos novos
passos para a continuidade da vida acad�mica deveriam ser discutidos com seu
orientador.
