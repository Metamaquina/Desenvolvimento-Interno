%% Modelo criado por R\'egis (c) 2011
%% http://latexbr.blogspot.com
%% twitter: @rg3915
%% Todos os caracteres foram convertidos para codigos TeX
%% para que possa ser lido em qualquer plataforma.
\documentclass[a4paper]{report} %padrao letterpaper, 10pt
\usepackage[utf8]{inputenc}
\usepackage[brazil]{babel}
\usepackage{amsfonts,amssymb,graphicx,enumerate}
\usepackage[centertags]{amsmath}
% Configuracoes de pagina
\usepackage[lmargin=3cm,rmargin=3cm,tmargin=3cm,bmargin=3cm]{geometry}
% Layout da pagina
\usepackage{hyperref}
\hypersetup{pdfpagelayout=SinglePage, % ou TwoPageLeft
    colorlinks=true,
    pdftitle={Exemplo de Relat\'orio},
    pdfauthor={R\'egis da Silva}}

%*******************************************************
% Definindo novos teoremas
\usepackage{theorem}
\newtheorem{teo}{Teorema}[chapter] % usa contador associado ao capitulo
\newtheorem{cor}[teo]{Corol\'ario} % usa mesmo contador do teorema
\newtheorem{lem}[teo]{Lema} % usa mesmo contador do teorema
\newtheorem{prop}[teo]{Proposi\c{c}\~ao} % usa mesmo contador do teorema
\newtheorem{axi}[teo]{Axioma} % usa mesmo contador do teorema
\theorembodyfont{\normalfont\upshape} % agora, nao eh mais fonte italico
\newtheorem{defn}[teo]{Defini\c{c}\~ao} % usa mesmo contador do teorema
\newtheorem{ex}{Exemplo}[chapter] % usa contador associado ao capitulo
\newtheorem{exerc}{Exercício}[chapter] % usa contador associado ao capitulo

% Ambiente para demonstracao que coloca quadrado no final, usando \rule
% Nota: \rule{largura}{altura} produz um retagulo preto.
\newenvironment{dem}[1][Demonstra\c c\~ao]{\textbf{#1:}\

}  {\hfill\rule{1ex}{1ex}}
%*******************************************************
% Definindo novos comandos
\providecommand{\sin}{} \renewcommand{\sin}{\hspace{2pt}\textrm{sen}}
\newcommand{\R}{\mathbb{R}} %simbolos de numeros reais
%*******************************************************
\title{Exemplo de Relat\'orio no \LaTeX}
\author{R\'egis da Silva}
\date{2011}    %para ocultar a data digite: \date{ }
%*******************************************************
\begin{document}    %Inicio do documento
\maketitle  %cria o titulo na capa

\tableofcontents %Sumario
%-------------------------------------------------------
\chapter{Primeiro cap\'itulo}
\label{chap_primeiro} %rotulo que pode ser usado em referencia cruzada

\section{Primeira Se\c c\~ao}
\label{sec_primeiro}

% ambiente definido no teorema. O que esta entre [] eh opcional
\begin{defn}[TFC]
Seja $f:[a,b] \to \R$ uma fun\c c\~ao integr\'avel e $F$ sua primitiva. Ent\~ao

\[
  \int_a^b {f(x)dx}  = F(b) - F(a)
\]
\end{defn}

Lista enumerada:

\begin{enumerate}[a)]
 \item primeiro item;
 \item segundo item;
 \item terceiro item.
\end{enumerate}

Lista n\~ao enumerada:

\begin{itemize}
 \item primeiro item;
 \item segundo item;
 \item terceiro item.
\end{itemize}

%-------------------------------------------------------
\chapter{Segundo cap\'itulo}
\label{chap_segundo}

\section{Segunda Se\c c\~ao}
\label{sec_segundo}

\subsection{Subse\c c\~ao}
\label{subsec_nome}

\begin{teo}
Exemplo de teorema.
\end{teo}

\begin{ex}
Exemplo de equa\c c\~ao matem\'atica com m\'ultiplas linhas.

\[
\begin{gathered}
  5x - 10 = 0 \hfill \\
  5x = 10 \hfill \\
  x = 2 \hfill \\ 
\end{gathered} 
\]

Para equa\c c\~oes mais complexas leia \href{http://ctan.tche.br/help/Catalogue/entries/voss-mathmode.html}{mathmode.pdf} e \href{http://latexbr.blogspot.com/2010/11/editor-de-equacoes-online.html}{Equa\c c\~oes online}.
\end{ex}

% Referencias bibliograficas
\begin{thebibliography}{99}
\bibitem{Stewart} STEWART, James. {\sl C\'alculo.} Vol. 1. S\~ao Paulo: Pioneira, 2006.

\bibitem{Oetiker}
OETIKER, Tobias. Et. Al. {\sl Introdu\c c\~ao ao {\LaTeXe}.}
\url{http://mirrors.ctan.org/info/lshort/portuguese-BR/lshortBR.pdf},2001.

\bibitem{Tantau}
TANTAU, Till. {\sl The TikZ and PGF Packages.}
\url{http://ctan.tche.br/graphics/pgf/base/doc/generic/pgf/pgfmanual.pdf}, 2010.

\bibitem{LaTeX}
\url{http://latexbr.blogspot.com}
\end{thebibliography}

\addcontentsline{toc}{chapter}{Refer\^encias Bibliogr\'aficas}

\end{document}  %Fim do documento